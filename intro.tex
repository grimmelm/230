In the scholarly literature on intermediary liability, economic claims are common -- but mostly informal. They are policy arguments, not testable propositions. It is not clear when two claims conflict, or when they can coexist. Indeed, it is often not even clear whether two authors are making the same claim, or different claims.

We do not propose to resolve any of these disputes. Instead, we aim to clarify the terms of the debate. In this paper, we recast arguments about intermediary liability into a common language -- the language of microeconomics. We give an economic model of intermediary liability, with equations and diagrams. We see six significant benefits from having such a model.

First and most fundamentally, modeling promotes communication. A suitable model can serve as a common framework for scholars to compare and contrast arguments. Our taxonomy of liability regimes reduces the at-times bewildering array of arguments about the proper scope of intermediary liability into a (we hope) orderly structure that makes it straightforward to see how different claims relate.

Second, modeling promotes intuition. A good model can bring out the consequences of a course of conduct, or make plain why parties behave the way that they do. There are several common patterns in intermediary-liability law that have simple and vivid expressions in our model.

Third, modeling promotes visualization. We have attempted to provide a simple and memorable visual shorthand for every moving part in our model and every interesting effect of a legal regime. For example, we hope that even you take nothing else away from this paper, you will have a clear visual sense for why a platform might either overmoderate or undermoderate even in the absence of liability.

Fourth, modeling promotes rigor. The process of writing down a model forces one to make implicit assumptions explicit. The process of reasoning through a model's consequences forces one to examine each claimed effect closely. In the course of working through our own model, we learned a lot about how arguments for and against intermediary immunity work, and this paper conveys some of what we have learned.

Fifth, modeling promotes proof. Given a set of assumptions, it is possible to show rigorously whether particular conclusions follow. We demonstrate, for example, that under our assumptions, strict liability consistently results in overmoderation. Of course, the real world is not required to comply with a proof about a model. But proofs like these make models more useful, because they help pin down the predictions the model actually makes.

And sixth, modeling promotes empiricism. This is not an econometric paper; it uses no datasets and very few numbers. But a model like ours helps pin down which econometric questions to ask. We hope that it provides a roadmap for future empirical work.

This paper has six Parts, not counting this introduction and a brief conclusion. Part \ref{sec:background} surveys  previous work in this space. Part \ref{sec:model} describes the model in formal detail. Parts \ref{sec:undermoderation} and \ref{sec:overmoderation} analyze a variety of liability regimes in detail.  Part \ref{sec:laws} shows how various current and proposed laws map on to those liability regimes.  \autoref{sec:extensions} discusses possible extensions of the model.