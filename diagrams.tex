% TikZ macros
\usetikzlibrary{arrows.meta, intersections, patterns}

\newenvironment{econ}[2]
{
  \begin{figure}[h]
	  \centering
  %\caption{#1}
  \label{#2}
  \begin{tikzpicture}[scale=1]
}
{
  \end{tikzpicture}
  \end{figure}
}

\NewEnviron{pgfecon}[2]
{
  \begin{figure}[h]
  \centering
  \caption{#1}
  \label{#2}
  \begin{tikzpicture}[scale=.75]
  \begin{axis}[
	  axis lines* = middle,
	  xmin = -.2,
	  xmax = 10,
	  width=12cm,
	  ymin = -1,
	  ymax = 6,
	  height=8cm,
	  xtick=\empty,
	  ytick=\empty,
	  clip=false
  ]
  \path[name path=axis] (0,0) -- (10,0);
  \BODY
  \end{axis}
  \end{tikzpicture}
  \end{figure}
}



\newcommand{\drawaxes}[2] { 
	\draw[name path = xaxis, -|] (-0.2,0) -- (10,0) node[below]{}; 
	\draw[name path = yaxis, ->] (0,-0.2) -- (0,5.5) node[left]{} ;
}


\newcommand{\harmplot} {
  \addplot [-, name path = harmline,
  domain = 0:10,
  samples = 250] {5}
  node[very near start, above]{$h$};
}

\newcommand{\scurveplot}[1] {
  \addplot [-, name path = lambda, thick,
  domain = 0:10,
  samples = 250] { \x < 1 ? 0 : 
        (\x < 5 ? (\x - 1)^ 2 / 6.4 : 
        (\x < 9 ? 5 - (9-\x)^2 / 6.4  :
        5
        ))}  node[right]{#1};
}

\newcommand{\hplot}{\scurveplot{$h(x)$}}

\newcommand{\lambdaplot} {\scurveplot{$h\lambda(x)$}}

\newcommand{\lambdaline} {
 \addplot [-, name path = lambda, thick, domain = 0:10, samples = 250] { \x /2 )}  node[right]{$h\lambda(x)$};
}


\newcommand{\testplot}[1] {
\addplot [-, name path = profit, thick,
domain = 0:10,
samples = 250] {#1}
node[very near end, left]{$t(x)$};
}
\newcommand{\profitplot} {
 \addplot [-, name path = profit, thick,
domain = 0:10,
samples = 250] { \x < 1.5 ? 2.5 : 
				  (\x < 6.5 ? 2.5 - (\x - 1.5)^2 / 15 : 
				  .8333 - (\x - 6.5) / 1.625)}
node[very near end, left]{$p(x)$};
}

\newcommand{\welfareplot} {
 \addplot [-, name path = profit, thick,
domain = 0:10,
samples = 250] { \x < 1.5 ? 3.5 : 
				  (\x < 6.5 ? 3.5 - (\x - 1.5)^2 / 18 : 
				  2.111 - (\x - 6.5) / 1.875)}
node[very near end, above]{$s(x)$};
}


\newcommand{\plotline}[3] {
\addplot [-, name path = #1, dotted,
domain = 0:10,
samples = 250] {#2}
node[at start, left]{#3};
}

\newcommand{\plotvalue}[4] {
\addplot [-, name path = #1, thick,
domain = 0:10,
samples = 250] {
   \x < 1.5 ? #2 : 
   (\x < 6.5 ? #2 - (\x - 1.5)^2 / #3 : 
   #2 - (25 / #3) - (10 * (\x - 6.5)) / #3 }
node[right]{#4};
}

\newcommand{\plotpartialvalue}[5] {
\addplot [-, thick, color=#5,
domain = #3:#4,
samples = 250] {
 \x < 1.5 ? #1 : 
 (\x < 6.5 ? #1 - (\x - 1.5)^2 / #2 : 
 #1 - (25 / #2) - (10 * (\x - 6.5)) / #2 }
}

\newcommand{\harmfunction}[1] {
	\draw[name path=harm, thick] (0,0) to (1,0) parabola (5,2.5) parabola[bend at end] (9,5) to (10,5) node[right]{#1};
}

\newcommand{\harmline}[1] {
	\draw[name path=harm, thick] (0,5) -- node[above]{#1} (10,5);
}

\newcommand{\revenueline}[1] {
	\draw[name path=revenue, thick] (0,1.5) to (10,1.5) node[right]{#1};
}

\newcommand{\revenuecurve}[1] {
	\draw [name path=revenue] (0,3) to (2, 3) parabola (6,1.75) to (10, -1) node[right]{#1};
}

\newcommand{\dropline}[3]{
  \draw[dashed, thin] (#1, #2) -- (#1, 0) node[below]{#3};
}

\newcommand{\xe} {
	\draw[dashed, thin] (4.1,1.5) -- (4.1,0) node[below]{$x^e$}; 
}

\newcommand{\xexhat} {
	\draw[dashed, thin] (4.1,1.5) -- (4.1,0) node[below]{$\hat{x} = x^e$}; 
}


\newcommand{\xhat}[1]{
   \draw[dashed, thin] (#1,1.5) -- (#1,0) node[below]{$\hat{x}$}; 
}

\newcommand{\spilloverfunction}[1] {
	  \draw [name path=welfare] (0,4.25) to (2, 4.25) parabola (6,3) to (10, 0.25) node[right]{#1};
}

\newcommand{\spilloverline}[1] {
	\draw[thick] (0,2.5) to (10,2.5) node[right]{#1};
}

% \NewDocumentCommand\drawarrows{O{0.5}mmO{-latex}mm}{
% \def\Step{#1}
% \pgfmathsetmacro{\Second}{#5+\Step}
% \foreach \Value [count=\xi] in {#5,\Second,...,#6}
%   {
%   \path[overlay,name path=line\xi] 
%     (\Value,100) -- (\Value,-100);
%   \path[name intersections={of=#2 and line\xi,by={1\xi}}];
%   \path[name intersections={of=#3 and line\xi,by={2\xi}}];
%   \ifdim#5pt<\Value pt\relax
%     \ifdim\Value pt<#6pt\relax
%     \draw[#4]
%       (1\xi) -- (2\xi); 
%   \fi\fi
%   }
%  }