\label{sec:background}



Economic models of liability start from the pioneering work of \cite{brown1973toward} and are further developed by many important contributions, including \cite{shavell1980strict}, which are summarized in \cite{shavell2009accident}. 
In the economic analysis, the goal of the liability system is to promote efficiency, which means minimizing the total social costs.  
The focus is on the effect of different liability rules, mainly strict liability and the negligence rule, on incentives for taking precaution to reduce risk. 
One of the key insights is that if only injurers influence risks, both strict liability and the negligence rule can induce the injurers to take optimal care.  

% injurers are firms
The economic theory of liability extends to the case of product liability where a manufacturer or seller might be held liable for harm caused by a defective or unsafe product. 
This requires us to take into account the relationship between liability, market price, and the profit-maximizing production of the firms. 
Important contributions include \cite{hamada1976liability}, \cite{polinsky1980strict}, \cite{landes1985positive}, \cite{polinsky2010uneasy}, and \cite{daughety2013economic} provides an excellent review. 
One of the key insights of this literature is whether and how to impose liability depends on the characteristics of the product and the information of the consumers.  
For example, a strict liability rule would be more appropriate when it is difficult to test for product safety.
In general, strict liability is more efficient than the negligence rule, as it either results in prices that induce optimal production or induces consumers to purchase the optimal quantity. 

Economic literature on intermediary liability, however, is still at its infancy stage.  
\cite{hua2021holding} extends the product liability framework to a two-sided platform that enables interactions between sellers and buyers. By either raising price or investing in screening, the platform wants to keep the safe sellers while deterring the harmful sellers. The authors argue that whether to impose liability on the platform is a judgement proof problem. If the sellers have deep pocket then intermediary immunity is optimal. If instead the sellers are judgement proof, then intermediary liability is necessary and imposing the residual liability on the platform improves social welfare.  
\cite{buiten2020rethinking} examine the liability of online hosting platforms for user-generated content, specifically in the context of the European Union's Digital Single Market strategy. They argue that the current EU liability framework is outdated and inadequate for dealing with the challenges of online content moderation. The absence of ``Good Samaritan'' protection in the EU e-Commerce Directive creates perverse incentives for platforms not to monitor online activity, thus undermining self-regulation. 
They propose a three-tiered approach, in which platforms are responsible for different types of content depending on their level of control over the content. 
This literature so far focus on comparing intermediary immunity with strict liability. In this paper, we also examines alternative liability rules that are popular in the legal debate, including negligence, must-carry rule, liability upon notice, and safe harbor. 