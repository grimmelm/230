\label{sec:background}

The literature on the economic analysis of law and the legal scholarship on intermediary liability are both immense. In the former, we draw particularly on the tradition of formalizing liability rules started by John Prather Brown,\note{brown1973toward} and on the standard distinction between strict liability and negligence.\note{shavell1980strict; shavell1980strict; see generally shavell2009accident (summarizing literature)} In this literature, the usual goal of the liability system is to promote efficiency, which means minimizing total social costs. The focus is on the effect of different liability rules on incentives for taking precaution to reduce risk. One of the key insights is that if only injurers influence risks, both strict liability and negligence can induce the injurers to take optimal care.  

The economic theory of liability has been extended to the case of product liability where a manufacturer or seller might be held liable for harm caused by a defective or unsafe product. \note{e.g. hamada1976liability; polinsky1980strict;landes1985positive; polinsky2010uneasy; see generally daughety2013economic (summarizing literature)} This theory takes into account the relationship between liability, market price, and firms' profit-maximizing production. One of the key insights of this literature is that whether and how to impose liability depends on the characteristics of the product and the information of the consumers. For example, a strict liability rule would be more appropriate when it is difficult to test for product safety. In general, strict liability is more efficient than negligence, as it either results in prices that induce optimal production or induces consumers to purchase the optimal quantity. One of the key purposes of our model is to square this conclusion with the widespread argument in the legal literature that strict liability is particularly inappropriate for platforms.

The economic literature on online intermediary liability, however, is still in its infancy. The most detailed treatment is a student note by Matthew Schruers (now the president of the Computer and Communications Industry Association).\note{schruers2002history} He analyzes four legal regimes: negligence, notice-based liability, strict liability, and immunity in a model where the intermediary can vary its level of care. Although the moving parts in his model are different than ours, it is an illuminating analysis of the tradeoffs involed in imposing intermediary liability. His most trenchant insight is that notice reduces the effort required for the platform to achieve a given level of care,\note{schruers2002history at 237-39} an approach that informs our own information-based treatment of liability on notice. He also notes the essential parallel between liability on notice and strict liability and the chilling effect of strict liability on online speech. 

In a more recent article, Xinyu Hua and Kathryn Spier extend the product-liability framework to a two-sided platform that enables interactions between sellers and buyers.\note{hua2021holding} By either raising price or investing in screening, the platform wants to keep the safe sellers while deterring the harmful sellers. They argue that whether to impose liability on the platform is a judgement-proof problem. If the sellers have deep pocket then intermediary immunity is optimal. If instead the sellers are judgement-proof, then intermediary liability is necessary and imposing residual liability on the platform improves social welfare.\note{See also lefouili2022economics}



% justification and support for sec230 
Intermediary immunity, as codified by Section 230, has been credited for promoting the growth of the Internet.\note{chander2013law} Many authors argue that Section 230 was a response to concerns about the negative impact of lawsuits on online service providers, and that Section 230 strikes a balance between free speech and safety.\note{ehrlich2002commuications;ziniti2008optimal; sanchez2008web; kosseff2010defending}
Some observe that platforms do moderate in the absence of any liability. They argue that even intermediary immunity might lead to the over-removal of content by the platform (through user account termination, shadow banning, or ``collateral censorship'').\note{goldman2012online; wucollateral}

At the same time, other scholars criticize the current shape of Section 230.\note{see generally reidenberg2012section (surveying scholarly analyses of Section 230 and proposed reforms)} Danielle Citron has argued that online platforms facilitate and amplify harassment and hate speech.\note{citron2014hate} She and others argue that courts have given Section 230 an overly broad interpretation and that this broad immunity provides excessively strong incentives to allow or encourage online materials to go unmoderated.\note{benedict2008deafening; ferris2010communication} Consequently, the broad immunity fails to protect the victims of online abuse with no recourse against the platforms whose profit-maximizing business models facilitate the harmful activities.\note{see, e.g., citron2018problem; bartow2009internet}


There is substantial literature advocating the reform of Section 230, but they disagree on the appropriate form of intermediary liability. Some papers seek to refine the scope of intermediary immunity by taking a multi-factor approach. For example, some authors suggest that courts should consider the level of editorial control exercised by the platform and the harm caused by defamatory statements, when determining immunity in each case (though it is not clear how the court should weigh these different factors).\note{see, e.g., lee2004batzel; defterderian2009fair; browne2015losing } Other authors discuss strict liability for certain kinds of harms;\note{see, e.g., maccarthy2010payment (discussing payments industry)} in particular, Nancy Kim suggests treating intermediary liability like product liability.\note{kim2008imposing} And some authors suggest applying criminal liability to sites that have been facilitating and profiting from illegal activities.\note{radbod2010craigslist}

Other scholars propose forms of conditional immunity. Danielle Citron and Benjamin Wittes have argued that the platforms should enjoy immunity only if they could prove that they took reasonable efforts to address online abuse, and lawmakers should specify the obligations for this duty of care.\note{citronwittes; see also citron2022fix}  Doug Lichtman and Eric Posner propose a system of graduated response, in which ISPs would issue warnings and impose penalties on users who engage in infringing behaviors.\note{lichtman2006holding} They suggest a conditional immunity rule in which ISPs would be held liable for infringing content only if they fail to implement reasonable measures to prevent or deter infringement. Caitlin Hall proposes an immunity conditioned on the display of rating labels that alert internet users of the credibility of information posted on the sites.\note{hall2008regulatory}

Other authors compare the different immunity regimes of Section 230 and Section 512.\note{lemley2007rationalizing; duran2003hear; holmes2001making medenica2007does; manekshaw2005liability; band2002safe}  Those who highlight what they see as the advantages of the DMCA-style regime note the ability of the victims to prevent further harm through a notice-and-takedown system, the coordination of the ISPs in removing harmful materials, and the administrative ease of transition.  On the other hand, those who worry about applying a notice-and-takedown system to all user-generated content highlight the possible chilling effects on free speech.

Other authors compare the intermediary liability regime in the European Union with that in the United States. Daphne Keller examines the relationship and the tension between the online platforms' liability in Europe (e.g., the E-Commerce Directive) and the EU's General Data Protection Regulation (GDPR).\note{keller2018right}
Miriam Buiten, Alexandre De Streel, and Martin Peitz examine the EU's Digital Single Market strategy and argue that the current EU liability framework is inadequate for dealing with the challenges of online content moderation.\note{buiten2020rethinking} They claim that the absence of ``Good Samaritan'' protection in the EU e-Commerce Directive creates perverse incentives for platforms not to monitor online activity, thus undermining self-regulation. 

There is also a strong empirical literature on content moderation.\note{see generally keller2020facts (surveying information released by platforms and independent research)} Collectively, this research suggests that platforms tend to have a bias towards over-removal.\note{keller2015empirical}  In 2005, Jennifer Urban and Laura Quilter presented the first set of descriptive statistics on the notice and takedown process under DMCA Section 512.\note{urban2005takedown} They found that corporations and business entities were the primary senders of the notices, a majority of the notices were sent for competition purposes, one third of the notices were questionable regarding the validity of the copyright infringement claim, and very few individual users responded with a counter-notice. In a follow-up study, Urban, Joe Karaganis, and Brianna Schofield emphasized the role of automation in sending complaints, and compare how the automated notices differ from the manual notices by small rightholders.\note{urban2017everyday} More recently, Daniel Seng compiled a larger dataset and gave more detailed statistics, questioning the validity of many takedown notices, especially those generated by automated systems.\note{seng2020copyrighting} Jonathon Penney surveyed 1,296 panelists with hypothetical scenarios on receiving a takedown notice, indicating some chilling effects of the policy.\note{penney2019privacy} Respondents broadly reported being less likely in future not only to share the same content again, but also to share content they themselves had created (72\%). Only 34\% said they would send a counter-notice or challenge a takedown they believed was wrong or mistaken. 

