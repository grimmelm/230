\label{sec:background}



Economic models of liability start from the pioneer work of \cite{brown1973toward} and are further developed by many important contributions, including \cite{shavell1980strict}, which are summarized in \cite{shavell2009accident}. 
In the economic analysis, the goal of the liability system is to promote efficiency, which means minimizing the total social costs.  
The focus is on the effect of different liability rules, mainly strict liability and the negligence rule, on incentives for taking precaution to reduce risk. 
One of the key insights is that if only injurers influence risks, both strict liability and the negligence rule can induce the injurers to take optimal care.  

% injurers are firms
The economic theory of liability extends to the case of product liability where a manufacturer or seller might be held liable for harm caused by a defective or unsafe product. 
This requires us to take into account the relationship between liability, market price, and the profit-maximizing production of the firms. 
Important contributions include \cite{hamada1976liability}, \cite{polinsky1980strict}, \cite{landes1985positive}, \cite{polinsky2010uneasy}, and \cite{daughety2013economic} provides an excellent review. 
One of the key insights of this literature is whether and how to impose liability depends on the characteristics of the product and the information of the consumers.  
For example, a strict liability rule would be more appropriate when it is difficult to test for product safety.
In general, strict liability is more efficient than the negligence rule, as it either results in prices that induce optimal production or induces consumers to purchase the optimal quantity. 

Economic literature on intermediary liability, however, is still at its infancy stage. 
\cite{hua2021holding} extends the product liability framework to a two-sided platform that enables interactions between sellers and buyers. By either raising price or investing in screening, the platform wants to keep the safe sellers while deterring the harmful sellers. The authors argue that whether to impose liability on the platform is a judgement proof problem. If the sellers have deep pocket then intermediary immunity is optimal. If instead the sellers are judgement proof, then intermediary liability is necessary and imposing the residual liability on the platform improves social welfare.  
\cite{buiten2020rethinking} examine the liability of online hosting platforms for user-generated content, specifically in the context of the European Union's Digital Single Market strategy. They argue that the current EU liability framework is outdated and inadequate for dealing with the challenges of online content moderation. The absence of ``Good Samaritan'' protection in the EU e-Commerce Directive creates perverse incentives for platforms not to monitor online activity, thus undermining self-regulation. 
They propose a three-tiered approach, in which platforms are responsible for different types of content depending on their level of control over the content. 
This literature so far focus on comparing intermediary immunity with strict liability. In this paper, we also examines alternative liability rules that are popular in the legal debate, including negligence, must-carry rule, liability upon notice, and safe harbor. 




% justification and support for sec230 
Intermediary immunity, as codified by Section 230, has been applauded on the ground of promoting free speech and the growth of the internet, which generates positive externality to society. \cite{ehrlich2002commuications} recognizes that Section 230 was a response to concerns about the negative impact of lawsuits on online service providers, and argues that Section 230 strikes a balance between free speech and censorship, while still allowing for accountability in certain cases. This position is supported by some later papers (\cite{ziniti2008optimal}, \cite{sanchez2008web}, \cite{kosseff2010defending}).
Some authors notice that platforms do moderate in the absence of any liability. They argue that even intermediary immunity might lead to the over-removal of content by the platform (through user account termination, shadow banning, or ``collateral censorship''), and users should be granted greater opportunities to challenge the removal decisions (\cite{ciolli2008chilling}, \cite{goldman2012online}, \cite{wucollateral}). It is not clear though, what is the efficiency benchmark they have in mind and whether they under-estimate the harm on the internet.

% problem for sec230
Existing literature has pointed out the problems of Section 230 in its current shape. 
\cite{citron2014hate} examines the phenomenon of online harassment and hate speech that are facilitated and amplified by online platforms, and discusses the long-lasting consequences of the harm to the victims. 
Courts also tend to give Section 230 an overly broad interpretation (\cite{benedict2008deafening}, \cite{ferris2010communication}, and see the references cited in \cite{reidenberg2012section}). 
As argued by \cite{tushnet2010attention} and others, the broad immunity provides strong incentives to allow or encourage online materials to go unmoderated. Consequently, the broad immunity fails to protect the harmed and leaves victims of online abuse with no recourse against the platforms whose profit-maximizing business models facilitate the harmful activities (\cite{citron2018problem}, \cite{bartow2009internet}).


% reform for sec230 
There is substantial literature advocating the reform of Section 230, but they disagree on the appropriate form of intermediary liability.
Some papers seek to refine the scope of intermediary immunity by taking a multi-factor approach. For example, \cite{lee2004batzel}, \cite{defterderian2009fair} and \cite{browne2015losing} suggest that courts should consider the level of editorial control exercised by the platform and the harm caused by defamatory statements, when determining immunity in each case (though it is not clear how the court should weigh these different factors).
Alternatives to intermediary immunity have also been proposed. 
One group of papers argue for strict liability for certain harms. 
\cite{kim2008imposing} suggests treating intermediary liability as product liability. \cite{radbod2010craigslist} suggests applying criminal liability to the sites that have been facilitating and profiting from illegal activities. And the harsh punishment of strict liability seems to function well in the online payment industry (\cite{maccarthy2010payment}). 
Another group of papers propose different forms of conditional immunity.
\cite{lichtman2006holding} propose a system of graduated response, in which ISPs would issue warnings and impose penalties on users who engage in infringing behaviors. They suggest a conditional immunity rule in which ISPs would be held liable for infringing content only if they fail to implement reasonable measures to prevent or deter infringement.
\cite{hall2008regulatory} propose an immunity conditioned on the display of rating labels that alert internet users of the credibility of information posted on the sites. 
\cite{citronwittes} and \cite{citron2022fix} argue that the platforms should enjoy immunity only if they could prove that they took reasonable efforts to address online abuse, and lawmakers should specify the obligations for this duty of care. 



Both Section 230 and Section 512 set liability standards for ISPs that host user-generated content, and they take significantly different approaches. 
Section 512 notably ties two liability regimes together: liability upon notice and safe harbor for controlling the total harm. 
Some authors highlight what they see as the advantages of a DMCA-style liability and advocate for a unified intermediary liability regime for both copyright infringement and other kinds of third-party content (\cite{duran2003hear}, \cite{holmes2001making}, \cite{medenica2007does}, \cite{manekshaw2005liability}, \cite{band2002safe}). Among the advantages, they note the ability of the victims to prevent further harm through a notice-and-takedown system, the coordination of the ISPs in removing harmful materials, and the administrative ease of transition.  


The possible chilling effects on free speech are the worries on applying a notice-and-takedown system to all user-generated content.
And empirical evidence has been accumulating on the effect of DMCA (\cite{keller2015empirical}, \cite{keller2020facts}). Collectively, they appear to suggest an over-removal bias resulting from the procedure.   
\cite{urban2005takedown} presents the first set of descriptive statistics on the notice and takedown process based on the Lumen database. The authors found that corporations and business entities were the primary senders of the notices, a majority of the notices were sent for competition purposes, one third of the notices were questionable regarding the validity of the copyright infringement claim, and very few individual users responded with a counter-notice. 
In a follow-up study by \cite{urban2017everyday}, the authors emphasize the role of automation in sending complaints, and compare how the automated notices differ from the manual notices by small rightholders. 
\cite{seng2020copyrighting} compiles a larger dataset and shows more detailed statistics, questioning the validity of many takedown notices, especially those generated by automated systems.
\cite{penney2019survey} surveys 1,296 panelists with hypothetical scenarios on receiving a takedown notice, indicating some chilling effects of the policy. Respondents broadly reported being less likely in future not only to share the same content again, but also to share content they themselves had created (72\%). Only 34\% said they would counternotice or challenge a takedown they believed was wrong or mistaken. 

